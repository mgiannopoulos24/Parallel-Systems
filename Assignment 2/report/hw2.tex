\documentclass{article}
\usepackage{graphicx} 
\usepackage{amsfonts,amsmath,amssymb}
\usepackage{array}
\usepackage{tabularray}
\usepackage[utf8]{inputenc}
\usepackage[T1]{fontenc}
\usepackage{csquotes}
\usepackage{alphabeta}
\usepackage{url}
\usepackage{hyperref}
\usepackage{esint}

\renewcommand{\figurename}{Γράφημα}

\begin{document}
\begin{table}[ht]
    \begin{tblr}{
        @{}X[l, valign=b]X[c, valign=b]X[r, valign=b]@{}
    }

    \hline
    % First line, course info
    \SetCell[c=2]{l}{[ΘΠ04] Παράλληλα Συστήματα} & & {2024-25} \\ 
    \hline
    {} & {} & {} \\

    % Title
    \SetCell[c=3]{c}{ \Large \textbf{Εργασία 1 - Προγραμματισμός με Pthreads} } \\
    {} & {} & {} \\

    % Name Surname, Student ID
    \hline \\
    \SetCell[c=3]{c}{ \textbf{Ονοματεπώνυμο:} Μάριος Γιαννόπουλος } \\
    \SetCell[c=3]{c}{ \textbf{A.M.:} 1115200000032} \\
    \hline

    \end{tblr}
\end{table}
\section*{Γενικές Πληροφορίες}

\subsection*{Υπολογιστικό Σύστημα}
Όλο το έργο υλοποιήθηκε στο ίδιο υπολογιστικό περιβάλλον:
\begin{itemize}
    \item \textbf{Όνομα Υπολογιστικού Συστήματος:} Linux12
    \item \textbf{Επεξεργαστής:} Intel(R) Core(TM) i5-6500 CPU @ 3.20GHz
    \item \textbf{Αριθμός Πυρήνων:} 4
    \item \textbf{Λειτουργικό Σύστημα:} Linux Ubuntu 20.04.2 LTS
    \item \textbf{Έκδοση Μεταγλωττιστή:} gcc (Ubuntu 9.4.0-1ubuntu1~20.04.2) 9.4.0
\end{itemize}

\end{document}